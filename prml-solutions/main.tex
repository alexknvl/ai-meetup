\documentclass[paper=letter, fontsize=11pt]{scrartcl}
\usepackage[utf8]{inputenc}
\usepackage{fourier}
\usepackage[english]{babel}

% Hyper links
\usepackage{hyperref}

% Math packages
\usepackage{amsmath, amsfonts, amsthm}
\usepackage{enumitem}
\usepackage{bussproofs}
\usepackage{graphics, setspace}
\usepackage{dashrule}

% The mathrsfs package provides the script font.
\usepackage{mathrsfs}

\usepackage{xcolor, dot2texi, tikz}
% The tiks-cd package provides macros for easily writing commutative diagrams.
% For more info see [1].
% [1]: http://ctan.math.ca/tex-archive/graphics/pgf/contrib/tikz-cd/tikz-cd-doc.pdf
\usepackage{tikz-cd}

\usetikzlibrary{shapes,arrows}
% Allows customizing section commands
\usepackage{sectsty}
% Make all sections centered, the default font and small caps
\allsectionsfont{\centering \normalfont\scshape}

% Custom headers and footers
\usepackage{fancyhdr}
% Makes all pages in the document conform to the custom headers and footers
\pagestyle{fancyplain}
% No page header - if you want one, create it in the same way as the footers
% below
\fancyhead{}
% Empty left footer
\fancyfoot[L]{}
% Empty center footer
\fancyfoot[C]{}
% Page numbering for right footer
\fancyfoot[R]{\thepage}
% Remove header underlines
\renewcommand{\headrulewidth}{0pt}
% Remove footer underlines
\renewcommand{\footrulewidth}{0pt}
% Customize the height of the header
\setlength{\headheight}{13.6pt}

% Number equations within sections
% (i.e. 1.1, 1.2, 2.1, 2.2 instead of 1, 2, 3, 4)
\numberwithin{equation}{section}
% Number figures within sections
% (i.e. 1.1, 1.2, 2.1, 2.2 instead of 1, 2, 3, 4)
\numberwithin{figure}{section}
% Number tables within sections
% (i.e. 1.1, 1.2, 2.1, 2.2 instead of 1, 2, 3, 4)
\numberwithin{table}{section}

% Removes all indentation from paragraphs - comment this line for an
% assignment with lots of text
\setlength\parindent{0pt}
\setlength{\parskip}{0.5em}

% Create horizontal rule command with 1 argument of height
\newcommand{\horrule}[1]{\rule{\linewidth}{#1}}

% Add lcm operator.
\DeclareMathOperator{\lcm}{lcm}

% The chngcntr ("change counter") package is used here so that subsection
% numbers are written without the leading section number. This takes place in
% the subsection headings as well as the theorem environment numbering.
%
% Before:
% 1. Section
% 1.1. Subsection
% Problem 1.1.1. What is 1 + 1?
% Problem 1.1.2. What is 1 + 2?
%
% After:
% 1. Section
% 1. Subsection
% Problem 1.1. What is 1 + 1?
% Problem 1.2. What is 1 + 2?
% (these `\let`s are to work around a "Command \counterwithout already defined"
% for those of us with inferior LaTeX setups)
\let\counterwithout\relax
\let\counterwithin\relax
\usepackage{chngcntr}
\counterwithout{subsection}{section}

% The problem environment is a regular ams theorem environment with "Problem"
% text and some leading space to give some separation between the problems.
\theoremstyle{definition}
\newtheorem{problem-internal}{Problem}[subsection]
\newenvironment{problem}{
  \medskip
  \begin{problem-internal}
}{
  \end{problem-internal}
}

% The solution environment is a proof environment with the "solution" text as
% well as the following adjustments:
% - No indent on paragraphs;
% - A small amount of space between paragraphs.
%
% Note: The negative space at the beginning is to remove the space before the
% first paragraph in the solution.
\newenvironment{solution}{
  \begin{proof}[Solution]
  \vspace{-8px}
  \setlength{\parskip}{4px}
  \setlength{\parindent}{0px}
}{
  \end{proof}
}

% Renewing the \thesection command changes the section numbers to roman
% numerals. This matches the style of the Aluffi textbook.
%
% Before:
% 1. Section
% 1.1. Subsection
%
% After:
% I. Section
% I.1. Subsection
\renewcommand{\thesection}{\Roman{section}}

% Set with spacing padding for the curly braces.
\newcommand{\set}[1]{\left\{\,#1\,\right\}}
\newcommand{\id}{\mathrm{id}}
\newcommand{\im}{\mathrm{im}}
\newcommand{\Obj}{\mathrm{Obj}}
\newcommand{\Hom}{\mathrm{Hom}}
\newcommand{\abs}[1]{\left|#1\right|}
% Intuitive "from" command draws an arrow pointing left, A <- B reads "A from p"
\newcommand{\from}{\leftarrow}
\newcommand{\quotuniv}[1]{\overline{#1}}

\title{	%	TITLE SECTION
  % Thin top horizontal rule
  \horrule{0.5pt} \\[0.4cm]
  \huge Pattern Recognition and Machine Learning \\
  % Thick bottom horizontal rule
  \horrule{2pt} \\[0.5cm]
}
\author{Alexander Konovalov}
% Today's date or a custom date
\date{\normalsize\today}


\begin{document}
\maketitle % Print the title

\section*{Chapter 1}
\subsection*{Introduction}
\setcounter{subsection}{1}

% Problem 1.1
\begin{problem}
  \begin{align*}
    y(x, w) &= \sum_i w_i \phi_i(x) \\
    E(w)    &= \frac{1}{2} \sum_j \left(y(x_j, w) - t_j\right)^2
  \end{align*}
  Setting the derivative of $E$ to zero with respect to $w_i$:
  \begin{align*}
    \frac{\partial E}{\partial w_i} &=
      \frac{1}{2} \cdot 2 \sum_j \left(y(x_j, w) - t_j\right) \cdot \phi_i(x_j) \\
      &= \sum_j \phi_i(x_j) y(x_j, w) - \sum_j t_j \phi_i(x_j) \\
      &= \sum_j \phi_i(x_j) \sum_k w_k \phi_k(x_j) - \sum_j t_j \phi_i(x_j) \\
      &= 0
  \end{align*}

  Let's define $A_{ik} = \sum_j \phi_i(x_j) \phi_k(x_j)$ and $T_i = \sum_j t_j \phi_i(x_j)$, then:
  \begin{align*}
    &\sum_j \phi_i(x_j)\sum_k w_k \phi_k(x_j) = \sum_j t_j \phi_i(x_j) \\
    &\sum_k w_k \sum_j \phi_i(x_j) \phi_k(x_j) = T_i \\
    &\sum_k A_{ik} w_k  = T_i
  \end{align*}

  Which is equivalent to a matrix equation $\mathcal{A} w = \mathcal{T}$.
\end{problem}

% Problem 1.2
\begin{problem}
  \begin{align*}
    E(w)    &= \frac{1}{2} \sum_j \left(y(x_j, w) - t_j\right)^2 + \frac{\lambda}{2}\sum_k w_k^2
  \end{align*}
  Setting the derivative of $E$ to zero with respect to $w_i$:
  \begin{align*}
    \frac{\partial E}{\partial w_i} &=
      \sum_k A_{ik} w_k - T_i + \frac{\partial}{\partial w_i}\left[\frac{\lambda}{2}\sum_k w_k^2\right] \\
      &= \sum_k A_{ik} w_k - T_i + \lambda w_i \\
      &= \sum_k A_{ik} w_k - T_i + \lambda \sum_k \delta_{ik} w_k \\
      &= \sum_k (A_{ik} + \delta_{ik}) w_k - T_i \\
      &= \sum_k A^{*}_{ik} w_k - T_i \\
      &= 0
  \end{align*}
\end{problem}

% Problem 1.3
\begin{problem}
  \begin{align*}
    P(F) = \sum_B P(F | B) P(B)
  \end{align*}
  Substituting $F=a$ (apple):
  \begin{align*}
    P(F=a) &= P(F=a | B=r) P(B=r) + P(F=a | B=b) P(B=b) + P(F=a | B=g) P(B=g) \\
           &= \frac{3}{10} \cdot 0.2 + \frac{1}{2} \cdot 0.2 + \frac{3}{10} \cdot 0.6 \\
           &= \frac{6}{100} + \frac{10}{100} + \frac{18}{100} \\
           &= \frac{34}{100} \\
           &= 34\%
  \end{align*}

  WIP
\end{problem}

% Problem 1.4
\begin{problem}
  Assume that $x = g(y)$, a strictly monotonically increasing function.
  Then $dx = g'(y) dy$ and
  \begin{align*}
    p_x(x) dx &= p_y(y) dy \\
              &= p_x(g(y)) g'(y) dy \\
    p_y(y) &= p_x(g(y)) g'(y)
  \end{align*}

  Differentiating with respect to $y$:
  \begin{align*}
    p_y'(y) &= p_x'(g(y)) (g'(y))^2 + p_x(g(y)) g''(y)
  \end{align*}

  If $p_x'(x_0) = 0$ and $x_0 = g(y_0)$,
  \begin{align*}
    p_y'(y_0) &= p_x'(g(y_0)) (g'(y_0))^2 + p_x(g(y_0)) g''(y_0) \\
              &= 0 \cdot (g'(y_0))^2 + p_x(g(y_0)) g''(y_0) \\
              &= p_x(x_0) g''(y_0)
  \end{align*}

  This means that unless $g''(y_0)$ is zero, an extremum of $p_x(x)$ does not
  map to an extremum of $p_y(y)$.

  Is $g''(y_0)=0$ a sufficient condition for a maximum of $p_x(x)$
  to map to a maximum of $p_y(y)$? If $p_x(x)$ reaches a maximum at $x_0$ and
  $p_x(x)$ is sufficiently smooth, then $p_x''(x_0) \leq 0$.
  Let's find $p_y''(y)$:

  \begin{align*}
    p_y'(y) &= p_x'(g(y)) (g'(y))^2 + p_x(g(y)) g''(y) \\
    p_y''(y) &= p_x''(g(y)) (g'(y))^3 + 2 p_x'(g(y)) g'(y) g''(y) + p_x'(g(y)) g'(y) g''(y) + p_x(g(y)) g'''(y)
  \end{align*}
\end{problem}

\end{document}
